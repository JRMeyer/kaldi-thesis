\chapter{CD content}
\label{cha:cd_content}
The CD contains source code of projects developed, extended or modified as implementation part of this thesis.
The thesis texts describes my work on projects listed below:

\begin{itemize}
    \item Alex --- Alex Dialogue System Framework where I added following files and directories:
        \begin{itemize}
            \item \term{alex/components/asr/kaldi.py} --- ASR component interfacing \term{PyOnlineLatgenRecogniser}
            \item \term{alex/tools/kaldi/} --- Kaldi training scripts modified for Alex 
            \item \term{alex/applications/PublicTransportInfoCs/hclg/} --- Decoding graph (\term{HCLG}) scripts, and scripts for \acs{ASR} evaluation.
        \end{itemize}
    \item The Kaldi toolkit --- Speech recognition toolkit where I added directories:
        \begin{itemize}
            \item \term{src/onl-rec} --- Implementation of \term{OnlineLatgenRecogniser} and utilities 
            \item \term{src/pykaldi} --- Python wrapper \term{PyOnlineLatgenRecogniser} and utilities
            \item \term{egs/vystadial/s5} --- Training scripts for acoustic modelling\footnote{The same scripts were integrated into Kaldi svn trunk repository. However, the scripts are separated for Czech and English data. See \url{http://sourceforge.net/p/kaldi/code/HEAD/tree/trunk/egs/vystadial_cz/} and \url{http://sourceforge.net/p/kaldi/code/HEAD/tree/trunk/egs/vystadial_en/}}
            \item \term{egs/vystadial/online\_demo} --- Demos using using \term{OnlineLatgenRecogniser} and \term{PyOnlineLatgenRecogniser}.
        \end{itemize} 
    \item Pyfst --- Python wrapper of OpenFst, where I improved installation and addedd several simple functions. Note I forked the original pyfst library.
    \item Pykaldi-eval --- Repository for evaluation OnlineLatgenRecogniser written in IPython notebook. See interesting graphs.
    \item thesis.pdf
    \item Reference documentation for C++ code in \term{kaldi/src/onl-rec}.
    \item Reference documentation for Python code in \term{kaldi/src/pykaldi}.
    \item The reference documentation for my code in Alex.
    % \item Related papers --- papers where I am main author or co-author and are related to this work.
\end{itemize}
