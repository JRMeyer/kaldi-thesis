% !TEX root = main.tex
\chapter{Acoustic model training}
\label{cha:train}

This chapter presents new Kaldi acoustic modeling recipe for free Czech and English "\term{Vystadial}" data.
The recipe scripts were developed as part of this thesis, they are licensed under the Apache 2.0 license and are publicly available in Kaldi repository\footnote{http://sourceforge.net/p/kaldi/code/HEAD/tree/sandbox/oplatek2/egs/vystadial/}.
The \ac{AM} trained using this scripts can be used for both batch speech recognition with common Kaldi decoders and for our \term{OnlineLatgenRecognizer}, which performs on-line decoding described in Chapter~\ref{cha:decoder}.

The first Section~\ref{sec:data} describes used data. 
The chapter continues by~presenting the \acp{AM} training in~Section~\ref{sec:am_train}. 
Later, in~Section~\ref{sec:am_eval} we evaluate trained \acp{AM} and also compare them to generative \ac{HTK} \acp{AM} which are trained using state of art \ac{HTK} scripts.
Note, that details about launching the scripts and file system organisation can be found in Appendix~\ref{cha:train_scripts}.  
The description and references to methods \todo{whose} setup we describe below can be found in Section~\ref{sec:back_asr}.

\section{Vystadial acoustic data}
\label{sec:data}

The data were collected in Vystadial project\footnote{http://ufal.mff.cuni.cz/grants/vystadial} and are released under the Creative Commons Share-alike (CC-BY-SA~3.0) license. 
The Czech\footnote{Czech data: \url{http://hdl.handle.net/11858/00-097C-0000-0023-4670-6}}and English\footnote{English data: \url{http://hdl.handle.net/11858/00-097C-0000-0023-4671-4}} data are available online in Lindat repository\footnote{{http://lindat.mff.cuni.cz/repository/}}\footnote{A previous version of our training scripts is published with the data in Lindat repository and described in~work~\cite{korvas_2014}.}.

The English acoustic data consists of recorded phone calls among humans and \acl{SDS}, which was designed to provide the user with information on a suitable dining venue in the town.
Most of the data was spoken in American English.
The typical sentences recorded from users were queries for the dialogue system e.g.,
\begin{verbatim}
I NEED A CHINESE TAKE AWAY RESTAURANT IN THE CHEAP PRICE RANGE
I'M LOOKING FOR AN INTERNATIONAL RESTAURANT
I NEED TO FIND A PUB IT SHOULD ALLOW CHILDREN AND HAVE A TELEVISION
\end{verbatim}.

On the other hand, the Czech recordings were collected in three different ways\cite{korvas_2014}:
\begin{enumerate}
    \item using a free Call Friend phone service
    \item using the Repeat After Me speech data collecting process,
    \item from telephone interactions with the Alex \ac{SDS} in public transport domain.
\end{enumerate}

In Call Friend service native Czech speakers were invited to make free calls.
In Repeat After Me process volunteers called a number where they were asked to repeat 
sentences synthesized by a \ac{TTS}.

The user language differs significantly in dialogues with Alex system and the other two settings.
% Remember we focus on training \acp{AM} for \ac{SDS} Alex and its public transport domain.
The sentences in Alex public transport domain, as seen in the first paragraph, are shorter and contain noises.
The speech is spontaneous and proper names are frequently used.
In contrary, the other two recording tasks, as seen in the second paragraph, have much broader vocabulary with less named entities, and the ideas are expressed in longer sentences.

\begin{verbatim}
A DALŠÍ
_NOISE_
JO DĚKUJU MOC TO JSEM CHTĚL VĚDĚT
ZE ZASTÁVKY DEJVICKÁ
\end{verbatim}

\begin{verbatim}
PRYČ S TYRANY A ZRÁDCI VŠEMI
UTRHNE SI KVĚT Z KYTICE A ODCHÁZÍ
DYŤ TO JE HROŠÍ NEŽ ZVÍŘE
O LIBERALIZMU TEHDY NEBYLO ŘEČI
CO BY TAM S TEBOU DĚLALI
\end{verbatim}


The \acp{AM} for Czech are trained on acoustic data from all three very different domains, because there is only 2 hours of in-domain data available in the Alex's public transport domain.
The evaluation for Czech data in~Section~\ref{sec:results} is performed  on test set combined from all three domains.
The Czech test set is mixed according the proportion of the domains in training and development set.
The English \acp{AM} are trained and tested on the data collected from the Venue domain using \ac{SDS}.
The summary of audio sizes in training, development and test set are presented in Table~\ref{tab:audio}.
Both Czech and English orthographic speech transcriptions were transcribed by humans.

\begin{table}[hbp]
    \centering
    \begin{tabular}{lrrr}
        \hline
            dataset & audio & \# sents & \# words \\
        \hline
        \textbf{English} & & & \\
                train & 41:30 & 47,463 & 178,110 \\
                dev & 01:45 & 2,000 & 7,376 \\
                test & 01:46 & 2,000 & 7,772 \\
        \hline
        \textbf{Czech} & & & \\
                train & 15:25 & 22,567 & 126,333 \\
                dev & 01:23 & 2,000 & 11,478 \\
                test & 01:22 & 2,000 & 11,204 \\
        \hline
		\end{tabular}
    \caption{Size of the data: length of the audio (hours:minutes), number of sentences
        (which is the same as the number of recordings), number of words in the 
    transcriptions.\cite{korvas_2014}}
    \label{tab:audio}
\end{table}


% On contrary, in evaluation of \ac{ASR} in Alex described in~Chapter~\ref{cha:integration}
% we use only transcribed utterances from Alex recordings, 
% so we can measure accuracy of the speech recognition task in~the dialog system.



\section{Acoustic modelling recipe}
\label{sec:am_train}

In the recipe we search for the best non-speaker adaptive \acp{AM}. 
In this section, the explored methods and their settings are described, and the Section~\ref{sec:results} presents the results for both Czech and English data sets.

The acoustic modelling techniques focus on modelling the speech to word mapping, so the test utterances are decoded with the least error possible. 
For correctness the testing uses previously unseen utterances in training or development set, so the real conditions are well simulated.
The Figure~\ref{fig:am-deps} list all acoustic models trained in our recipe.
The advanced \ac{AM} is always initiated by audio alignments (respectively acoustic features alignments) obtained using simpler \ac{AM}.

In paragraphs below the organisation of acoustic model training is described. 
The used methods are listed in~Figure~\ref{fig:am-deps} and also the their hierarchy is illustrated.
The hierarchy shows that more advanced method typically reuses initial values from previously trained simpler \ac{AM}.

At first, a mono-phone model is trained from flat start using the MFCCs, $\Delta$ and $\Delta \Delta$ features.
We force-align the feature vectors to HMM states for phones in the corresponding transcriptions.
Secondly, We retrain the triphone \ac{AM} (\term{tri1a}).
One branch of experiments finishes by training \ac{MFCC} $\Delta + \Delta\Delta$ triphone \ac{AM} (\term{tri2a}). % force-aligned using \term{tri1a} \ac{AM}.

On contrary, the second branch instead of $\Delta + \Delta\Delta$ transformation uses \ac{LDA}+\ac{MLLT} to train \ac{AM} (\term{tri2b}).
Using the \ac{AM} \term{tri2b} three \acp{AM} are discriminatively trained using following objective functions:
\begin{enumerate}
    \item \acl{MMI}\cite{chow1990maximum}\footnote{Note the \ac{MMI} function is implemented as \acs{bMMI} with boosted parameter set to 0.}. The model \term{tri2b\_mmi}is train in four loops.
    \item \acl{bMMI}\cite{povey2008boosted}. The model \term{tri2b\_bmmi} is train in four loops with parameter 0.05.
    \item \acl{MPE}\cite{povey2003mmi}. The model \term{tri2b\_mpe} is also retrained in four loops.
\end{enumerate}

\begin{figure}[!htp]
    \begin{center}
    % Generated with LaTeXDraw 2.0.5
% Tue Feb 11 11:13:23 CET 2014
% \usepackage[usenames,dvipsnames]{pstricks}
% \usepackage{epsfig}
% \usepackage{pst-grad} % For gradients
% \usepackage{pst-plot} % For axes
\scalebox{1} % Change this value to rescale the drawing.
{
\begin{pspicture}(0,-1.0892187)(7.2534375,1.0892187)
\usefont{T1}{ptm}{m}{n}
\rput(0.44109374,0.40578124){mono}
\usefont{T1}{ptm}{m}{n}
\rput(2.1203125,0.36578125){tri1}
\usefont{T1}{ptm}{m}{n}
\rput(3.68625,0.8857812){tri2a}
\usefont{T1}{ptm}{m}{n}
\rput(3.6567187,-0.17421874){tri2b}
\usefont{T1}{ptm}{m}{n}
\rput(6.1254687,0.50578123){tri2b\_mmi}
\usefont{T1}{ptm}{m}{n}
\rput(6.215469,-0.21421875){tri2b\_bmmi}
\usefont{T1}{ptm}{m}{n}
\rput(6.1534376,-0.8542187){tri2b\_mpe}
\psline[linewidth=0.04cm,arrowsize=0.05291667cm 2.0,arrowlength=1.4,arrowinset=0.4]{->}(1.1071875,0.45578125)(1.7471875,0.43578124)
\psline[linewidth=0.04cm,arrowsize=0.05291667cm 2.0,arrowlength=1.4,arrowinset=0.4]{->}(2.4471874,0.55578125)(3.1871874,0.87578124)
\psline[linewidth=0.04cm,arrowsize=0.05291667cm 2.0,arrowlength=1.4,arrowinset=0.4]{->}(2.4871874,0.41578126)(3.2071874,-0.10421875)
\psline[linewidth=0.04cm,arrowsize=0.05291667cm 2.0,arrowlength=1.4,arrowinset=0.4]{->}(4.1471877,0.03578125)(5.1471877,0.5157812)
\psline[linewidth=0.04cm,arrowsize=0.05291667cm 2.0,arrowlength=1.4,arrowinset=0.4]{->}(4.2671876,-0.20421875)(5.1671877,-0.20421875)
\psline[linewidth=0.04cm,arrowsize=0.05291667cm 2.0,arrowlength=1.4,arrowinset=0.4]{->}(4.1471877,-0.30421874)(5.0871873,-0.82421875)
\end{pspicture} 
}

    \small{\begin{tabular}{lll}
    \hline
    Training method name & Script shortcut \\
    \hline
    Monophone & mono \\
    Triphone  & tri1 \\
    $\Delta + \Delta\Delta$ & tri2a  \\
    \acs{LDA}+\acs{MLLT} & tri2b  \\
    \acs{LDA}+\acs{MLLT}+\acs{MMI} & tri2b\_mmi \\
    \acs{LDA}+\acs{MLLT}+\acs{bMMI} & tri2b\_bmmi \\
    \acs{MPE} & tri2b\_mpe \\
    \hline
    \end{tabular}}
    \end{center}
    \caption{Training partial order among \ac{AM} in our training recipe}
    \label{fig:am-deps} 
\end{figure}

The acoustic models \term{mono}, \term{tri1}, \term{tri2a}
and \term{tri2b} are trained generatively.
The discriminative models \term{tri2b\_mmi}, \term{tri2b\_bmmi} and \term{tri2b\_mpe} yield better results than generative models, if enough data is available. 
See Figure~\ref{fig:partials}.
Note that the discriminative may over-fit to train data, so models from second or third retraining loop may provide better results in general case. 
However, we have not experienced such behaviour.
The discriminative methods use a \ac{LM} for improving its results over generative models by discriminating according their objective function.
In our setup an unigram \ac{LM} estimated on train set transcriptions was used for discriminative training methods.

The training scripts for Czech and English data differ only in using different phonetic dictionary and preprocessing the data using the dictionary, but the training itself remains exactly the same.
The default bigram and zerogram \acp{LM} for testing are built from orthographic transcriptions.
The bigram \ac{LM} is estimated from training data transcriptions. 
Consequently, in a test set may appear \acl{OOV} words.
The zerogram is extracted from a test set transcriptions.
The zerogram is a list of words with probabilities uniformly distributed, so it helps decoding just by limiting the vocabulary size.

% As we are collecting more and more acoustic data, the acoustic models improve and the language models generated by the scripts slightly change.
% Obviously, the changing \acp{LM} disallow fair evaluation of trained \acp{AM}, but our goal is not to change
In this work, we present results with fixed data sets as described in~Table~\ref{tab:audio}.
The bigram \ac{LM} contains 17433 unigrams and 79333 bigrams. The zerogram \ac{LM} is limited to 2944 words.

Furthermore in~Chapter~\ref{cha:integration}, we evaluate the trained Czech \acp{AM} on Public Transport Domain on different test set with a fine tuned \ac{LM} and best \ac{AM} from list in Figure~\ref{fig:am-deps}.
The best \ac{AM} is selected based on results in Section~\ref{sec:am_eval}.
% The best Czech \ac{AM} with the fined tuned \ac{LM} is used for speech recognition in a dialogues system Alex.


\subsubsection*{Setup for feature transformations}
We explore not only \ac{AM} training methods, but we also experiment with two feature transformation techniques.
Firstly, the $\Delta + \Delta\Delta$ triples the number of 13 \ac{MFCC} features by computing also first and second derivatives from \ac{MFCC} coefficients, resulting in 39 features per frame.

Secondly, the combination of \ac{LDA} and \ac{MLLT} is computed from 9 spliced frames consisting of 13 \ac{MFCC} features. 
The default context window of 9 frames takes four frames from left context and four frames from right context.
The \ac{LDA} and \ac{MLLT} feature transformation gains substantial improvement over $\Delta+\Delta\Delta$ transformation.
See Figure~\ref{fig:partials}.

\subsection*{Decoding setup}
% \label{sub:decoding_setup}
We described the training setup in sections above, and now we describe the setup for testing and evaluating the trained \acp{AM}.
For each trained \ac{AM} we used the same speech parametrisation and feature transformation as used for given \ac{AM} at training time.
We experiment with all trained \acp{AM} with both zerogram and bigram \ac{LM}.

The \ac{AM}, \acp{LM}, phonetic dictionary and few helper files are used to built \term{HCLG} decoding graph for each \ac{AM} and feature transformation combination.
The decoding graph is used for speech recognition on development and test test.
The parameters are set to default values; the exceptions are \term{beam=12.0, lattice-beam=6.0, max-active-states=14000} and \term{lmw} \acl{LMW}. 
The \term{lmw} is estimated on the development set and best value is used for testing.
The details about the parameters can be found in Section~\ref{sec:real-setup}.

The \term{gmm-latgen-faster} decoder is used for evaluation on testing data.
It generates a word level lattice for each utterance and the one best hypothesis is extracted from the decoded lattice and evaluated by \ac{WER} and \ac{SER} metrics against the reference transcription.

Note, we are able to exactly reproduce the results of \term{gmm-latgen-faster} decoder with our \term{OnlineLatgenRecogniser}, but the \term{gmm-latgen-faster} was used for evaluation in the scripts, so the Kaldi users do not have to install our extension.
% The speed of the decoding is not important at this moment. 

\section{Evaluation}
\label{sec:am_eval}

The experiments focus on comparing the quality of ASR hypothesis measured by~\ac{WER} on \acp{AM} trained by different methods.
We are not interested in absolute numbers since we model the language using a weak \ac{LM} focusing on the acoustic modeling.
By training only simple bigram \ac{LM} we let the \ac{AM} influence the recognition quality more significantly. 
The same motivation lead us to use zerogram \ac{LM} which just limits vocabulary in decoding task, and does not advice the decoding search more probable phrases as higher order \ac{LM} does.
Consequently, the best words are chosen among all hypotheses only by acoustic similarity.

We concentrate on acoustic modelling since we believe that; if two \acp{AM} $am_1$, $am_2$ are trained with the same weak $lm_{weak}$ and the first \ac{AM} gains lower \ac{WER} than the second one ($wer^{weak}_{1} <  wer^{weak}_{2}$), then in the same experiment just with richer \ac{LM} the first \ac{AM} will still gain lower \ac{WER} ($wer^{rich}_{1} <  wer^{rich}_{2}$).

Firstly, we show how the data size influence the quality of \acp{AM} measured by \ac{WER}.
Secondly,  the best results on full data is presented.
Finally in Subsection~\ref{sec:compare}, the best Kaldi results are compared against results obtained well-written \ac{HTK} scripts by Keith Vertanen and improved by Matěj Korvas \cite{korvas_2014} on the same Vystadial dataset.

\begin{figure}[!htp]
    \begin{center}
    \includegraphics[scale=0.7]{images/partial-zerogram.ps}
    \caption{Czech generative \acp{AM} performance based on different training size for acoustic models. The zerogram LM results in high WER, but allows as evaluate only acoustic modelling.}
    \label{fig:partials} 
    \end{center}
\end{figure}

The Figure~\ref{fig:everyn} describes how the amount of acoustic data influence the \ac{WER}.
We illustrate that even with small datasets like Vystadial the high quality \ac{AM} can be trained.
The WER decreases significantly if new data are added to small dataset, but \ac{WER} reduction becomes is very small between 50\% of data and full data.
One can also see that the $\Delta+\Delta\Delta$ feature transformation is clearly outperformed by \ac{LDA}+\ac{MLLT} setup on full data.
Note also that the monophone \ac{AM} is typically used for initialisation of triphone models and requires small portion of data to reach its limit.
The WER is rather high due to use of zerogram \ac{LM}.
We evaluate only generative \acp{LM} since we would have to fixed LM for discriminative methods and we do not have any obvious choice how to build one.

It may seem that more acoustic data is not needed for this domain, but discriminative training methods requires more training data, and with more transcribed data better \ac{LM} adaptation can be achieved.
The Figure~\ref{fig:partials_lm} shows the effect of in-domain data size for \ac{LM} on quality of speech decoding.
The \ac{AM} \term{tri2b\_bmmi} and decoding parameters were fixed and the experiments were performed with different \acp{LM} which differ only in training size used for their estimation. 
Note that this experiment was run by Ondřej Dušek and not as part of the training scripts\footnote{Ondřej Dušek used the scripts developed by us for Alex dialogue system for Public Transport Domain}.
The experiment was run on different test set from Public Transport Domain and the \acp{LM} were built also from that in-domain data.
\begin{figure}[!htp]
    \begin{center}
    \includegraphics[scale=0.7]{images/partial-lm-tri2b-bmmi.ps}
    \caption{Influence of in-domain text size of \ac{LM} on speech recognition quality. The \ac{AM} \term{tri2b\_bmmi} and parameters are fixed and only \ac{LM} training size varies.}
    \label{fig:partials_lm} 
    \end{center}
\end{figure}

To conclude first section, we continue to collect new acoustic data through dialogue system Alex because
\begin{itemize}
    \item the domain still changes due to new features and we need to update the \ac{LM},
    \item we can still improve the best discriminatively trained \ac{AM},
    \item and the speech recogniser is presumably more robust to new speakers.
\end{itemize}

\subsection{Results}
\label{sec:results}
In this section we present the results of different acoustic training methods and we choose the best non-speaker adaptive setup.
The Table~\ref{tab:best} presents \acp{AM} results. 

\begin{table}[h]
\centering
\begin{tabular}{lrrr}
    \toprule
            \theader{language/method}
            & \hphantom{rogram}\llap{\theader{zerogram}}
                            & \theader{bigram} 
                            & \theader{RTF} \\
    \midrule
            \theader{Czech} & & \\
                \hspace{2\tabindent}tri $\Delta+\Delta\Delta$
                &   70.7 &   56.6  & -1.0 \\
                \hspace{2\tabindent}tri LDA+MLLT
                &   68.2 &   53.9 & -1.0 \\
                \hspace{2\tabindent}tri LDA+MLLT+MMI
                &    65.3  &   49.5 & -1.0 \\
                \hspace{2\tabindent}tri LDA+MLLT+BMMI
                &    65.3  &   49.3 & -1.0 \\
                \hspace{2\tabindent}tri LDA+MLLT+MPE
                &    63.8  &   49.2 & -1.0 \\
    \midrule
        \theader{English} & \\
            \hspace{2\tabindent}tri $\Delta+\Delta\Delta$
            &   41.1 &   17.5 & -1.0 \\
            \hspace{2\tabindent}tri LDA+MLLT
            &   37.3 &   17.2 & -1.0 \\
            \hspace{2\tabindent}tri LDA+MLLT+MMI
            &  TODO &   12.0 &  -1.0 \\
            \hspace{2\tabindent}tri LDA+MLLT+BMMI
            &    TODO  & TODO & -1.0 \\
            \hspace{2\tabindent}tri LDA+MLLT+MPE
            &    TODO  & TODO & -1.0 \\
    \bottomrule
\end{tabular}
\caption{Word error rates for zerogram and bigram LM for different training triphone methods.
    The RTF was measured for bigram \ac{LM}.
    The `tri~$\Delta+\Delta\Delta$' row shows results for a generative model which is comparable to the model trained using the HTK scripts.
}
\label{tab:best}
\end{table}

The complexity of Czech data is clearly much larger than the complexity of English data.

\subsection[Kaldi and \acs{HTK} comparison]{Kaldi and previous \ac{HTK} results comparison} 
\label{sec:compare}

We present results for triphone \ac{AM} estimated using Baum-Welsch iterative training on zero-gram and bi-gram \acp{LM}.
The \term{HVite} \ac{HTK} decoder was used to perform the decoding with the same \acp{LM} as used in Kaldi scripts. 
The training procedure is further described in work~\cite{korvas_2014}.
We present 

\begin{table}[h]
  \centering
    \begin{tabular}{lrr}
    \toprule
            \theader{language/method} & \theader{zerogram} & \theader{bigram} \\
    \midrule
            \theader{Czech}& & \\
         \hspace{2\tabindent}tri $\Delta+\Delta\Delta$  & 64.5 & 60.4\\
        \midrule
      \theader{English}& & \\
           \hspace{2\tabindent}tri $\Delta+\Delta\Delta$  & 50.0 & 17.5 \\
        \bottomrule
  \end{tabular}
    \caption{Word error rates on test set obtained using HTK and either 
    a zerogram or a bigram LM. \cite{korvas_2014}}
    \label{tab:htk-results}
\end{table}

The results suggest that Kaldi achieves similar WER compared to HTK when 
using standard generative training methods and bigram LMs.
Using more advanced discriminative training methods, one can obtain 
a substantial decrease in WER.

The experiment using \ac{MFCC}, \ac{LDA} \& \ac{MLLT} and \ac{bMMI} discriminative training is
a state of the art set up for speaker independent speech recognition\cite{morbini2013asr} and outperforms \ac{HTK} models.

The \ac{WER} on the Vystadial English data is lower than 20\% for discriminative methods which is reasonable,
given the broad but limited domain.
The WER on the Vystadial Czech data from mixed domains are rather high, presumably because of lack of acoustic data for such large \todo{17000} vocabulary.
Nevertheless, the training scripts for the Czech data are very important since there are no other Czech acoustic data available\footnote{According our knowledge.}.


% Note, we tried to simulate the~\ac{HTK} settings in the~Kaldi experiment~\ref{tab:htk_like}.
% We choose for all experiments the~parameters according the~\ac{HTK} scripts.
% The most important parameters are the~maximum Gausians number and the~number of \acl{PDF}.
% \todo{pdf numbers and max gausians why 19200? and how did I compute it?} 

% \todo{In the experiment~\ref{tab:htk_like} we used the available options for reproducing the~\ac{HTK} like generation
% of \ac{MFCC} features.
% }
% 
% To conclude, we find out that Kaldi recognition toolkit is capable of training acoustic models with comparable quality 
% like \ac{HTK} toolkit using maximum likelihood training. In addition, Kaldi has rich set of tools for discriminative training, 
% which outperforms the maximum likelihood methods.
% In Chapter~\ref{cha:decoder} we will describe Kaldi decoders, which are convenient for real-time usage, 
% and which also does not loose the ability to produce quality ASR hypothesis.
