% !TEX root = main.tex
\chapter{Integration into a Dialog System}
\label{cha:integration}
This chapter briefly discuss the technical details of interfacing
a decoder from a dialog system.
In addition, in~Section~\ref{sec:python_c_interface} we present our project specific issue.
Namely, interfacing C/C++ code from Python.


\section{Dialog system architecture} 
\label{sec:dialog_system_architecture}
The Alex dialog system is developed in Python language and consist of~six major components. 
\begin{enumerate}
    \item Voice Activity Detection (VAD)
    \item Automatic Speech Recognition (ASR) 
    \item Spoken Language Understanding (SLU)
    \item Dialog Manager (DM)
    \item Natural Language Generation (NLG)
    \item Text To Speech (TTS)
\end{enumerate}
The~Alex dialog system has a~speech to speech user interface. The~system interacts with the user in {\it turns}. During a~single turn the dialog system waits for a~user spoken input, process the speech and generates the reply.
The~data~flow during single turn is depicted in~Figure~\ref{fig:dialog_system}.

The integration task consist of:
\begin{itemize}
    \item Building a~Python interface for a~C++ Kaldi decoder.
    \item Define the input and output interface for the Kaldi decoder - wrapped by Automatic Speech Recognition (ASR) unit in Alex.
\end{itemize}
 The ASR unit takes output of~the VAD component and provides input for the SLU unit. 
 %The input and output interfaces are chosen according the real time needs of~the VAD unit and the SLU unit.

\begin{figure}
    \begin{center}
    % Generated with LaTeXDraw 2.0.5
% Wed Apr 09 16:50:40 CEST 2014
% \usepackage[usenames,dvipsnames]{pstricks}
% \usepackage{epsfig}
% \usepackage{pst-grad} % For gradients
% \usepackage{pst-plot} % For axes
\scalebox{1} % Change this value to rescale the drawing.
{
\begin{pspicture}(0,-3.66)(14.64,3.66)
\definecolor{color685}{rgb}{0.11764705882352941,0.14901960784313725,0.8901960784313725}
\definecolor{color523}{rgb}{0.027450980392156862,0.25882352941176473,0.8431372549019608}
\definecolor{color34}{rgb}{0.7843137254901961,0.0784313725490196,0.23529411764705882}
\psellipse[linewidth=0.04,dimen=outer](5.55,1.51)(0.83,0.69)
\usefont{T1}{ptm}{m}{n}
\rput(5.5046873,1.425){VAD}
\psellipse[linewidth=0.04,dimen=outer](8.97,2.61)(0.83,0.69)
\usefont{T1}{ptm}{m}{n}
\rput(8.942186,2.525){ASR}
\psellipse[linewidth=0.04,dimen=outer](12.51,1.15)(0.83,0.69)
\usefont{T1}{ptm}{m}{n}
\rput(12.455312,1.065){SLU}
\psellipse[linewidth=0.04,dimen=outer](12.61,-1.51)(0.83,0.69)
\usefont{T1}{ptm}{m}{n}
\rput(12.567813,-1.595){DM}
\psellipse[linewidth=0.04,dimen=outer](5.49,-1.27)(0.83,0.69)
\usefont{T1}{ptm}{m}{n}
\rput(9.001249,-2.315){NLG}
\psellipse[linewidth=0.04,dimen=outer](9.03,0.0)(5.61,3.66)
\usefont{T1}{ptm}{m}{n}
\rput(5.4315624,-1.355){TTS}
\psellipse[linewidth=0.04,dimen=outer](9.03,-2.23)(0.83,0.69)
\usefont{T1}{ptm}{m}{n}
\rput(9.0,0.24){\Huge Dialogue System}
\psline[linewidth=0.04cm,arrowsize=0.05291667cm 2.0,arrowlength=1.4,arrowinset=0.4]{->}(9.76,2.38)(11.9,1.6)
\psline[linewidth=0.04cm,arrowsize=0.05291667cm 2.0,arrowlength=1.4,arrowinset=0.4]{->}(6.36,1.76)(8.18,2.42)
\psline[linewidth=0.04cm,arrowsize=0.05291667cm 2.0,arrowlength=1.4,arrowinset=0.4]{->}(12.54,0.46)(12.56,-0.86)
\psline[linewidth=0.04cm,arrowsize=0.05291667cm 2.0,arrowlength=1.4,arrowinset=0.4]{->}(11.84,-1.74)(9.9,-2.32)
\psline[linewidth=0.04cm,arrowsize=0.05291667cm 2.0,arrowlength=1.4,arrowinset=0.4]{->}(8.14,-2.18)(6.3,-1.52)
\pscircle[linewidth=0.04,dimen=outer](0.98404694,2.1548762){0.6451238}
\psline[linewidth=0.04cm](1.0,1.0)(0.92,-0.16)
\psline[linewidth=0.04cm](0.98193634,1.0577894)(0.0,1.84)
\psline[linewidth=0.04cm](0.98193634,1.0432783)(2.0694895,2.1461184)
\psline[linewidth=0.04cm](0.9072397,-0.23489322)(1.46,-1.72)
\psline[linewidth=0.1cm,linecolor=color523,linestyle=dashed,dash=0.16cm 0.16cm,arrowsize=0.05291667cm 2.0,arrowlength=1.4,arrowinset=0.4]{->}(2.28,1.86)(4.46,1.5)
\psline[linewidth=0.1cm,linecolor=color523,linestyle=dashed,dash=0.16cm 0.16cm,arrowsize=0.05291667cm 2.0,arrowlength=1.4,arrowinset=0.4]{<-}(2.32,-0.52)(4.36,-0.88)
\usefont{T1}{ptm}{m}{n}
\rput(3.0603125,-1.055){\Large \color{color685}Speech}
\usefont{T1}{ptm}{m}{n}
\rput(3.0803125,1.345){\Large \color{color685}Speech}
\usefont{T1}{ptm}{m}{n}
\rput(6.8231244,2.605){\color{color34}Speech}
\usefont{T1}{ptm}{m}{n}
\rput(11.20875,2.605){\color{color34}Text/Lattice}
\usefont{T1}{ptm}{m}{n}
\rput(12.445,-0.435){\color{color34}Semantic meaning}
\usefont{T1}{ptm}{m}{n}
\rput(11.265624,-2.255){\color{color34}Action}
\usefont{T1}{ptm}{m}{n}
\rput(6.899688,-2.175){\color{color34}Text}
\psline[linewidth=0.04cm](0.8872397,-0.23489322)(0.1,-1.62)
\psline[linewidth=0.04cm](0.36193633,2.0032785)(1.72,2.76)
\psdots[dotsize=0.12](1.34,2.28)
\rput{48.07518}(1.8088533,-0.49558815){\psarc[linewidth=0.04](1.46,1.78){0.12}{0.0}{180.0}}
\psline[linewidth=0.04cm](1.0368464,1.4389827)(1.0031536,1.1610173)
\end{pspicture} 
}

    \caption{Single turn in Alex dialog system}
    \label{fig:dialog_system} 
    \end{center}
\end{figure}

In our dialog system we experiment with different outputs of~a~speech decoder in Spoken Language Understanding unit. 
The decoder for our system should be able to generate {\it n-best lists}, {\it lattices} and {\it confusion networks} output.
% Ideally, the decoder for our system would be able to generate {\it n-best lists}, {\it lattices} and {\it confusion networks}.

%   d) output of~the decoder will be standard lattices 
%     (both phone and word)
%   e) compute posterior lattices (both phone and word),
%   f) provide confusion networks for these lattices
%   g) measure teh quality of~the lattices, depth, oracle error rate    

% section integrate_kaldi_decoder_into_vystadial_framework (end)


% \subsection{Development of~a~real time decoder}
% \label{sub:kaldi_rt_decoder}
% In the our dialog system we need above all a~real time decoder for the Kaldi toolkit. At first, we will identify what prevents the current Kaldi decoders from being used as real time decoders. 
% 
% Namely, we will explore problems with latency and suggest potential speed improvements e.g.\ approximations, use of~\ac{GPU}, and so on. Based on the experiments we will suggest the optimal setting for the real time use of~Kaldi decoder.
% 
% % section development_of_real_time_asr_decoder_for_the_kaldi_toolkit (end)

We will also compare and contrast possibilities of~different output formats for each of~tested decoders. The~memory consumption and the~stability of~decoders will be also observed. 



% section dialog_system_architecture (end)

\section{Python/C++ interface} 
\label{sec:python_c_interface}
Why I choose cffi and not traditional swig/ boost what so ever solution?

% section python_c_interface (end)

\todo{Contrast and compare with google. Julius?}

\section{Software development in Open Source community} 
\label{sec:software_development_in_open_source_community}
\todo{ Shortly say that the source code will be merged into Kaldi and also promote Alex which is going to be on Github}


% section software_development_in_open_source_community (end)

% chapter integration_into_dialog_system (end)
