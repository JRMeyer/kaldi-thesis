% !TEX root = main.tex
\chapter{Real time decoder}
\label{cha:decoder}

\todo{Ten development vyhodit. Ty uz nemas cas na zadny vyvoj. Radeji dej do evaluation, ze rozeberes nejenom presnost 
ale i rychlosty dekoderu, a rohodnes ktery dekoder z Kaldi je vhodny a jake nastaveni potrebuje pro RT performance.}

\todo{We used OpenJulius for decoding lattices and normal outputs -- describe again the problems}
\todo{Mention the socket! API of OpenJulius}
\todo{dat tam analyza -> future improvements}
\todo{udelat real time experiment }
\todo{srovnani Julius prohledavani   pulka stranky}
\section{Online decoder original architecture} 
\label{sec:kaldi_decoder_architecture}
This section describes the architecture online decoder developed by Kaldi team.
The online decoder was used as starting point for building \ac{FST} real time decoder for our dialog system.

\todo{Describe architecture}
% section kaldi_decoder_architecture (end)

\section{Improving online decoder} 
\label{sec:improve}
We will present how we overcome the limitations of Kaldi online decoder.

% section online_decoder_improvements (end)


\section[Comparison of real time decoders]{Comparison of real time decoders in Alex dialog system} 
\label{sec:comparison_of_real_time_decoders_in_alex_dialog_system}
Further, in this section we will briefly compare the main qualities of used decoders.
Namely, we will compare: 
\begin{itemize}
    \item Kaldi online decoder - Acoustic models were trained by training scripts for Kaldi described at~\ref{cha:models}.
    \item OpenJulius - OpenJulius uses in our setup acoustic models trained by \ac{HTK}. 
    \item HDecode - HDecode is \ac{HTK} decoder and it decodes from acoustic models trained by \ac{HTK}.
\end{itemize}


% section comparison_of_real_time_decoders_in_alex_dialog_system (end)
% chapter real_time_decoder (end)
