\chapter{Experiments setup} 
\label{cha:setup}

\todo{Organizace experimentu neni az tak dulezita, dulezite jsou pouzite metody}
\todo{popsat presne jake LM jsem pouzil, jake vysledky budu reportovat}
\ml{Vystadial recipe}
\small{The Kaldi toolkit has codified form of training scripts with predefined file system structure. 
We created new recipe called {\it kaldi-Vystadial-recipe}\/ or just {\it Vystadial recipe}. 
In~the Vystadial recipe directory there is "s5" subdirectory where all the~recipe files are located. 

The Kaldi toolkit uses small utilities which can be combined together by Perl or Bash scripts.
The scripts used for training are described below. 
The structure of the Vystadial recipe folder can be seen in~Figure~\ref{fig:s5_dir}.
\begin{itemize}
    \item The~"run.sh" script launches the training and its evaluation.
    \item Scripts in "local" directory handle data preparation and other tasks specific for Vystadial recipe.
    \item Scripts in "utils" and "steps" are the common utilities for all recipes. 
    \item The "cmd.sh", "path.sh" and files in "conf" directory store settings.
    \item The "data" directory is generated by "run.sh" script and contains the transformed input data.
    \item The "exp" directory is generated by "run.sh" script and stores the trained models and evaluation results.
\end{itemize}

\tiny\begin{verbbox}
kaldi
|-- egs
|   |-- kaldi-Vystadial-recipe
|   |   |-- s5
|   |   |   |-- cmd.sh
|   |   |   |-- conf
|   |   |   |   |-- decode.config
|   |   |   |   |-- mfcc.conf
|   |   |   |   |-- train_conf.sh
|   |   |   |-- data
|   |   |   |   |-- test 
|   |   |   |   |-- train
|   |   |   |   |-- lang
|   |   |   |-- exp
|   |   |   |-- local
|   |   |   |   |-- backup.sh
|   |   |   |   |-- results.py
|   |   |   |   |-- save_check_conf.sh
|   |   |   |   |-- score.sh
|   |   |   |   |-- Vystadial_data_prep.sh
|   |   |   |   |-- Vystadial_format_data.sh
|   |   |   |   |-- Vystadial_prepare_dict.sh
|   |   |   |-- path.sh
|   |   |   |-- run.sh
|   |   |   |-- steps -> ../../wsj/s5/steps
|   |   |   |-- utils -> ../../wsj/s5/utils
\end{verbbox}
\normalsize

\begin{figure}[!htp]
\centering \theverbbox \caption{\small{Kaldi Vystadial recipe file system structure}}
\label{fig:s5_dir}
\end{figure}

Running one experiment means that several acoustic models are trained from the same data and with the same settings. 
Each of the acoustic models is trained by different training method and it is evaluated separately.

Each experiment in Vystadial recipe use the~supplied Vystadial dataset, which is split into training and testing data.
Each of the training method from~Table~\ref{tab:disc_train} is trained on data from "$data/train$" directory
and the method is evaluated on data from "$data/test$" directory.


\section{GmmLatgenWrapper}}
\label{sec:gmmlatgenwrapper}
The best settings are \todo{This and that and we used the AM trained with this parameters
and build LM with that parameters}

\subsubsection*{GmmLatgenWrapper parameters}
Full list of GmmLatgenWrapper parameters and their default values invoked by running
the {\it GmmLatgenWrapper.Setup(bad\_params)}\/ with bad parameters:
\lstinputlisting[columns=flexible, basicstyle=\scriptsize, breaklines=true]{./snippets/pygmmlatgenwrapper-params.txt}



% subsubsection full_list_of_gmmlatgenwrapper_parameters_and_their_default_values (end)



% section gmmlatgenwrapper_ (end)
