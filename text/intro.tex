% !TEX root = main.tex
\chapter{Introduction}
\label{chap:intro}

% todo few words
% 
% stezejni integrace - dialogovy system
%     jedna~cesta~OpenJulius - one-best OK, lattice malo kvalitni
% 
% srovnani OpenJulius
%     Kdo dela~lepsi lattice
%     kouknout se na~OpenJulius scripty
%     a~porovnat je
% 
% uvolneni: Budou volne dostupny a~pak clanek a~tutorial
% o datech o trenovacich sktriptech a~o decodovani pres 2 jazyky

A spoken dialog is the most intuitive way of~communication among people. Nowadays, the spoken dialog is becoming popular 
choice of~communication even for human-device interaction. The quality of~a~dialog largely depends on the quality 
of~speech recognition, because the~reasoning and the~reply is based on the recognized speech. 

In this work, we improve speech recognition in the dialog system called Alex. 
The Alex dialog system is developed at \ac{UFAL} \ac{MFF}. One of~the application of~Alex is a~public phone call service, where a~user can ask Alex for 
a~piece of advice about Prague public transport.
% about navigation in Prague public transport.

\section{The problem} 
\label{sec:why}
% The Alex dialog system and this thesis was partly funded by the Ministry of~Education, Youth and Sports
% of~the Czech Republic under the grant agreement LK11221 for~the~{\it Vystadial project}\/ and core research of~Charles University in Prague.
% Let us cite the main goals of~the~{\it Vystadial project}\cite{jurcicek2012vystadial}:
% \begin{quote}
%     \begin{itemize}
%         \item Study and improve methods for learning statistical models used in dialogue systems. 
%         \item Create software infrastructure for remote training and evaluation.
%         \item Release the developed dialogue system under open-source license.
%         \item Create publicly available corpus of~audio recordings, transcriptions and semantic annotations for training dialogue systems.
%     \end{itemize}
% \end{quote}

The Alex dialog system has been using the \ac{HTK} toolkit\cite{young94htk} to train acoustic models and 
the OpenJulius\cite{lee2009julius} decoder for real time decoding. Nevertheless, the used set up has several flaws.

One of~the goals is to release the Alex dialog system under 
Apache License, Version 2.0\footnote{\url{http://www.apache.org/licenses/LICENSE-2.0}}. 
By using \ac{HTK} toolkit we cannot publish Alex with the Apache License, 
because we need to modify the code of the decoders we are using. 

The \ac{HTK} decoders are not real-time decoders\cite{yao2010practical}. 
We used the~OpenJulius decoder for real-time decoding. OpenJulius can output except classic textual transcriptions of~audio
also lattices.Lattice is a convenient output format for a dialog system and will be described in~Section~\ref{sub:lattice}.

Unfortunately, we have experienced software instability using OpenJulius decoder.  
In addition, OpenJulius seems to be hard to be patched or otherwise improved due to its coding style. 
Currently, it seems, that the stability problems around OpenJulius are not going to be resolved in near feature.

The Kaldi\cite{povey2011kaldi} framework solves most of the drawbacks of~\ac{HTK} and OpenJulius. 
It is released under Apache License, Version 2.0. The same license we want to use. 
The Kaldi framework is also cleanly written and has a~responsive developer community. 
There are decoders implemented in Kaldi, which already support lattices. 
Currently, the Kaldi framework is mostly used for experimenting and it is not meant for real time usage. 
However, in August 2012\footnote{The changes were introduced 
by \href{https://sourceforge.net/p/kaldi/code/1259/}{svn commit 1259}.} Kaldi team published a~version of~an online decoder. 
In this thesis we hope to improve and integrate the online decoder into Alex dialog system for a~real time use.
The implemented decoder will be evaluated against current set up with OpenJulius.

% section why_introducing_new_decoder_and_toolkit_for_training_acoustic_models_ (end)

\section{The goals of~the~thesis} 
\label{sec:goals}
Let us introduce the goals of~this thesis. Each of~the~goals is described in one of subsections below.
At first, we introduce the goal of training acoustic model, which is the precondition for satisfying next goals.
Later, we prepare the real-time decoder and at the end, we integrate the decoder in the~dialog system. 

\subsection{Training acoustic models} 
\label{sub:training_kaldi_acoustic_models}
Every modern continuous speech recognition engine requires two pre trained components, an~acoustic model and 
a~language model. We will focus on finding the best acoustic model for the~Kaldi toolkit. 

The \ac{HTK} acoustic models have been used together with OpenJulius. 
We will develop training scripts for acoustic models using Kaldi toolkit with comparable results as
were obtained by using \ac{HTK} toolkit. 

% subsection training_kaldi_acoustic_models (end)
 

\subsection{Preparing real-time decoder} 
\label{sub:compare_rt}
Having an~acoustic model with~sufficient quality for the~data will allow us
compare and contrast Kaldi decoders. We will search for a convenient decoder,
which can be easily integrated to the~dialog system as real-time decoder.

We will select the most convenient Kaldi decoder and if needed, we will modify the decoder as necessary. 
We will suggest potential speed improvements e.g.\ approximations, use of \ac{GPU}.
% will balance speech recognition accuracy for the speed and select the best setting for the~use case.
In addition, we will consider different possibilities for output formats for different decoders.
% subsection preparing_real_time_decoder (end)

\subsection{A~real-time decoder integration} 
\label{sub:integration}
The final goal of this thesis is integration of selected real-time decoder or decoders into the~dialog system Alex.
We already use the~OpenJulius decoder in Alex. The~OpenJulius decoder is interfaced with other the rest of Alex
through an {\it ASR Python module}. We will integrate the Kaldi decoder into the {\it ASR module}\/ as well.

Furthermore, we will develop Python bindings with nice \ac{API} which wraps the C++ Kaldi decoder.\footnote{In contrast, 
the~OpenJulius is currently interfaced through subprocesses and Unix sockets, which is quite problematic interface} 
At the end, we will contrast and compare the~implementation with the set up using OpenJulius and \ac{HTK} 
acoustic models.

% subsection decoder integration (end)

% section goals (end)

\section{Outline of~the thesis} 
\label{sec:outline_of_the_thesis}
In~Chapter~\ref{cha:background} we describe components of a~speech recognition system.  
Except the standard introduction to speech recognition system we introduce Finite State Automata~framework,
which is used by Kaldi toolkit. In~Chapter~\ref{cha:training} we describe the acoustic 
models needed by the~Kaldi decoder. 
In addition, we contrast and compare acoustic models trained by~\ac{HTK} and Kaldi. 
Chapter~\ref{cha:decoder} presents the Kaldi real time decoder embedded into our dialog system.
We discuss the architecture of~the decoder, its properties. We also distinguish what is the original work done by 
the~Kaldi team and what are our improvements. At~the end of~Chapter~\ref{cha:decoder} 
we evaluate the qualities of~the developed decoder.
The~Chapter~\ref{cha:integration} explains how we integrated C++ decoder in our dialog system written in Python.
Finally, the~Chapter~\ref{cha:conclusion} summarises the thesis and finishes with future research directions.

% section outline_of_the_thesis (end)
