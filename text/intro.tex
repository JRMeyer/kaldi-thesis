% !TEX root = main.tex
\chapter{Introduction}
\label{cha:intro}

Spoken dialogue is the~most intuitive form of~communication among people. 
Spoken dialogue systems allow people to communicate with machines in an natural way.
The~quality of~a~dialogue with~a~spoken dialogue system largely depends on speech recognition because the~machine's understanding of the~user's utterance is based on the~recognised speech. 

% This work presents a new Kaldi\cite{povey2011kaldi} speech recogniser, acoustic modeling scripts, and evaluation of the developed recogniser in the~dialogue system Alex\cite{ptics2014url}.

\acf{ASR} in a~spoken dialogue system converts speech to text so that the dialogue system is able to extract semantic meaning from the~text.
Dialogue systems are able to exploit multiple alternative text hypotheses in addition to the single best hypothesis which is always returned. 
State-of-the-art speech recognisers are able to extract multiple hypotheses in real time.
We prefer extracting multiple hypotheses in form of~a~word lattice because it effectively represents multiple hypotheses.

The~Alex dialogue system had used the~\ac{HTK} toolkit\cite{young94htk} and the~OpenJulius\cite{lee2009julius} lattice speech recogniser to train acoustic models and to decode lattices in real time respectively. 
Unfortunately, our project members were experiencing crashes of OpenJulius during extracting lattices.
Fixing OpenJulius's complicated source code seemed unrealistic due to lack of documentation and community support.

HTK and OpenJulius was replaced by the~Kaldi toolkit\cite{povey2011kaldi} because its speech recognisers are able to produce high-quality lattices and are sufficiently fast\footnote{So far, 
    the~Kaldi developers focused on improving acoustic model training. 
    However, In August 2012 a Kaldi team published a~demo version of~an on-line one best hypothesis speech recogniser.} 
for real-time recognition.\cite{povey2012generating}
In addition, the~Kaldi toolkit deploys modern training recipes, is actively maintained, and is distributed under the~permissive Apache 2.0 license\footnote{\url{http://www.apache.org/licenses/LICENSE-2.0}}.
We still need to implement a speech recogniser which supports incremental speech processing, prepare acoustic modeling scripts and evaluate the developed recogniser, so that the Kaldi toolkit can be used in Alex dialogue system.

\section{The~goals of~the~thesis} 
\label{sec:goals}
The~goals of the~thesis are presented in order as will be implemented:
\begin{enumerate}
    \item \acp{AM} will be trained to evaluate the~new recogniser.
    \item The~new recogniser will be developed so its Python wrapper can be deployed into our dialogue system Alex.
    \item Finally, we will integrate the~recogniser into our Alex \ac{SDS} written in Python and evaluate its performance.
\end{enumerate}

\subsection{Training acoustic models} 
\label{sub:training_kaldi_acoustic_models}
A~speech recogniser requires two pre trained components, an~\acl{AM} and a~\acl{LM}. 
We focus on finding the~best \acl{AM} for the~Kaldi toolkit. 
% The~\acl{LM} is changed dependently on targeted domain.

We will develop acoustic modeling scripts using the~Kaldi toolkit\cite{povey2011kaldi} with such quality, that quality of trained \acp{AM} could be compared with the~\acp{AM} trained with the~\ac{HTK} toolkit. 
The~scripts will be developed for Czech and English acoustic data.

\subsection{Development real-time speech recogniser} 
\label{sub:compare_rt}

We should modify Kaldi speech recogniser in order to allow incremental speech recognition.
The~resulting incremental interface should be as simple as possible yet allow state-of-the-art performance.
In addition, we will implement such speech parametrisation and feature transformation preprocessing, so high-quality acoustic models can be used.
Finally, we should compute the~posterior probabilities of the~word lattice representing multiple \ac{ASR} hypotheses.

% The~process of incremental speech recognition is represented by a~single instance of \term{OnlineLatgenRecogniser} class, which provides simple interface to speech recognition.

In addition, we may suggest potential speed improvements e.g.\ approximations, use of \ac{GPU} or \ac{DNN}\cite{vesely2013sequencediscriminative}.

\subsection[Integration into Alex \acs{SDS} framework]{Integration into Alex \acl{SDS} framework} 
\label{sub:integration}
We should develop a~thin wrapper which efficiently exposes the~speech recognition interfaces to Python.
% We should also interface the~lattices which are the~output of the~Kaldi recogniser to Python.
The~resulting recogniser should be integrated into Alex \ac{SDS} and the~decoding parameters should be tuned to obtain best performance.
The~evaluation of the~speech recognition setup is an~important part of the~integration.

\section*{Thesis outline} 
In~Chapter~\ref{cha:background} we introduce a~fundamental theory of speech recognition for related areas to our work.
In Sections~\ref{sec:back_htk} and~\ref{sec:back_julius} we describe alternatives to Kaldi speech recognition toolkit. 
At the~end of the~chapter, we present OpenFST framework which allows the~Kaldi library effectively implement many standard speech recognition operations. 
To obtain high-quality \aclp{AM}, we develop training scripts for Czech and English data described in~Chapter~\ref{cha:train}. 
In addition, we compare acoustic models trained by Kaldi and previously used~\ac{HTK} toolkit. 
Chapter~\ref{cha:decoder} presents in detail the~new Kaldi real-time recogniser and discuss its on-line properties.
We distinguish the~original work done by the~Kaldi team and our improvements. 
Then in Chapter~\ref{cha:integration}, we describe deployment of the~real-time recogniser into dialogue system Alex, we suggest evaluation criteria and also evaluate the~integrated recogniser accordingly.
Finally, Chapter~\ref{cha:conclusion} summarises the~thesis and concludes with future research directions.
