% !TEX root = main.tex
\chapter{Introduction}
\label{chap:intro}

% todo few words
% 
% stezejni integrace - dialogovy system
%     jedna~cesta~OpenJulius - one-best OK, lattice malo kvalitni
% 
% srovnani OpenJulius
%     Kdo dela~lepsi lattice
%     kouknout se na~OpenJulius scripty
%     a~porovnat je
% 
% uvolneni: Budou volne dostupny a~pak clanek a~tutorial
% o datech o trenovacich sktriptech a~o decodovani pres 2 jazyky

A spoken dialog is the most intuitive way of~communication among people. Nowadays, the spoken dialog is becoming popular choice of~communication even for human-device interaction. The quality of~a~dialog largely depends on the quality of~speech recognition, because the~reasoning and the~reply is based on the recognized speech. 

In this work, we hope to improve speech recognition in the dialog system developed at \acs{UFAL} \acs{MFF} \acs{UK}, called Alex. One of~the application of~our dialog system Alex is a~public phone call service, where a~user can ask Alex about navigation in Prague public transport.

\section{The problem} 
\label{sec:why}
% The Alex dialog system and this thesis was partly funded by the Ministry of~Education, Youth and Sports
% of~the Czech Republic under the grant agreement LK11221 for~the~{\it Vystadial project}\/ and core research of~Charles University in Prague.
% Let us cite the main goals of~the~{\it Vystadial project}\cite{jurcicek2012vystadial}:
% \begin{quote}
%     \begin{itemize}
%         \item Study and improve methods for learning statistical models used in dialogue systems. 
%         \item Create software infrastructure for remote training and evaluation.
%         \item Release the developed dialogue system under open-source license.
%         \item Create publicly available corpus of~audio recordings, transcriptions and semantic annotations for training dialogue systems.
%     \end{itemize}
% \end{quote}

The Alex dialog system has been using the \ac{HTK} toolkit\cite{young94htk} to train acoustic models and the OpenJulius\cite{lee2009julius} decoder for real time decoding. Nevertheless, the used set up has several flaws.

\todo{ TO MUCH ABOUT LICENSE, very few about Julius INSTABILITY}
One of~our main goals is to release the Alex dialog system under Apache License, Version 2.0\footnote{\url{http://www.apache.org/licenses/LICENSE-2.0}}. By using \ac{HTK} toolkit we cannot publish Alex with the Apache License, because we need to modify the code of the decoders we are using. 

The \ac{HTK} decoders are not real-time decoders\cite{yao2010practical}. We used the~OpenJulius decoder for real-time decoding.

Unfortunately, we have experienced software instability using OpenJulius decoder. We used OpenJulius for decoding lattices and the OpenJulius decoderIn addition, OpenJulius seems to be hard to be patched or otherwise improved due to its coding style. Currently, it seems, that the stability problems around OpenJulius are not going to be resolved in near feature.
The~OpenJulius decoder, which is released under revised BSD style license.

The Kaldi\cite{povey2011kaldi} framework solves the drawbacks of~\ac{HTK} and OpenJulius. It is released under Apache License, Version 2.0. The same license we want to use. The Kaldi framework is cleanly written and has a~responsive developer community. The decoders implemented in Kaldi already support lattices and confusion networks generation. Currently, the Kaldi framework is mostly used for experimenting and it is not meant for real time usage. However, in August 2012\footnote{The changes were introduced by \href{https://sourceforge.net/p/kaldi/code/1259/}{svn commit 1259}.} Kaldi team published a~testing version of~an online decoder. In this thesis we hope to improve and integrate the online decoder into Alex dialog system for a~real time use.

% section why_introducing_new_decoder_and_toolkit_for_training_acoustic_models_ (end)

\section{The goals of~the~thesis} 
\label{sec:goals}
Let us introduce the goals of~this thesis. Each of~the~goals is described in one of subsections below. Note, that the goals are described in top down approach.
At first, we introduce the goal of integration the final real-time decoder into our~dialog system, and later we specify subgoals needed for the integration.
At~last, we promise to evaluate the work done.

\subsection{A~real-time decoder integration} 
\label{sub:integration}
The Alex dialog system is developed in Python language and consist of~six major components. 
\begin{enumerate}
    \item Voice Activity Detection (VAD)
    \item Automatic Speech Recognition (ASR) 
    \item Spoken Language Understanding (SLU)
    \item Dialog Manager (DM)
    \item Natural Language Generation (NLG)
    \item Text To Speech (TTS)
\end{enumerate}
The~Alex dialog system has a~speech to speech user interface. The~system interacts with the user in {\it turns}. During a~single turn the dialog system waits for a~user spoken input, process the speech and generates the reply.
The~data~flow during single turn is depicted in~Figure~\ref{fig:dialog_system}.

The integration task consist of:
\begin{itemize}
    \item Building a~Python interface for a~C++ Kaldi decoder.
    \item Define the input and output interface for the Kaldi decoder - wrapped by Automatic Speech Recognition (ASR) unit in Alex.
\end{itemize}
 The ASR unit takes output of~the VAD component and provides input for the SLU unit. 
 %The input and output interfaces are chosen according the real time needs of~the VAD unit and the SLU unit.

\begin{figure}
    \begin{center}
    % Generated with LaTeXDraw 2.0.5
% Wed Apr 09 16:50:40 CEST 2014
% \usepackage[usenames,dvipsnames]{pstricks}
% \usepackage{epsfig}
% \usepackage{pst-grad} % For gradients
% \usepackage{pst-plot} % For axes
\scalebox{1} % Change this value to rescale the drawing.
{
\begin{pspicture}(0,-3.66)(14.64,3.66)
\definecolor{color685}{rgb}{0.11764705882352941,0.14901960784313725,0.8901960784313725}
\definecolor{color523}{rgb}{0.027450980392156862,0.25882352941176473,0.8431372549019608}
\definecolor{color34}{rgb}{0.7843137254901961,0.0784313725490196,0.23529411764705882}
\psellipse[linewidth=0.04,dimen=outer](5.55,1.51)(0.83,0.69)
\usefont{T1}{ptm}{m}{n}
\rput(5.5046873,1.425){VAD}
\psellipse[linewidth=0.04,dimen=outer](8.97,2.61)(0.83,0.69)
\usefont{T1}{ptm}{m}{n}
\rput(8.942186,2.525){ASR}
\psellipse[linewidth=0.04,dimen=outer](12.51,1.15)(0.83,0.69)
\usefont{T1}{ptm}{m}{n}
\rput(12.455312,1.065){SLU}
\psellipse[linewidth=0.04,dimen=outer](12.61,-1.51)(0.83,0.69)
\usefont{T1}{ptm}{m}{n}
\rput(12.567813,-1.595){DM}
\psellipse[linewidth=0.04,dimen=outer](5.49,-1.27)(0.83,0.69)
\usefont{T1}{ptm}{m}{n}
\rput(9.001249,-2.315){NLG}
\psellipse[linewidth=0.04,dimen=outer](9.03,0.0)(5.61,3.66)
\usefont{T1}{ptm}{m}{n}
\rput(5.4315624,-1.355){TTS}
\psellipse[linewidth=0.04,dimen=outer](9.03,-2.23)(0.83,0.69)
\usefont{T1}{ptm}{m}{n}
\rput(9.0,0.24){\Huge Dialogue System}
\psline[linewidth=0.04cm,arrowsize=0.05291667cm 2.0,arrowlength=1.4,arrowinset=0.4]{->}(9.76,2.38)(11.9,1.6)
\psline[linewidth=0.04cm,arrowsize=0.05291667cm 2.0,arrowlength=1.4,arrowinset=0.4]{->}(6.36,1.76)(8.18,2.42)
\psline[linewidth=0.04cm,arrowsize=0.05291667cm 2.0,arrowlength=1.4,arrowinset=0.4]{->}(12.54,0.46)(12.56,-0.86)
\psline[linewidth=0.04cm,arrowsize=0.05291667cm 2.0,arrowlength=1.4,arrowinset=0.4]{->}(11.84,-1.74)(9.9,-2.32)
\psline[linewidth=0.04cm,arrowsize=0.05291667cm 2.0,arrowlength=1.4,arrowinset=0.4]{->}(8.14,-2.18)(6.3,-1.52)
\pscircle[linewidth=0.04,dimen=outer](0.98404694,2.1548762){0.6451238}
\psline[linewidth=0.04cm](1.0,1.0)(0.92,-0.16)
\psline[linewidth=0.04cm](0.98193634,1.0577894)(0.0,1.84)
\psline[linewidth=0.04cm](0.98193634,1.0432783)(2.0694895,2.1461184)
\psline[linewidth=0.04cm](0.9072397,-0.23489322)(1.46,-1.72)
\psline[linewidth=0.1cm,linecolor=color523,linestyle=dashed,dash=0.16cm 0.16cm,arrowsize=0.05291667cm 2.0,arrowlength=1.4,arrowinset=0.4]{->}(2.28,1.86)(4.46,1.5)
\psline[linewidth=0.1cm,linecolor=color523,linestyle=dashed,dash=0.16cm 0.16cm,arrowsize=0.05291667cm 2.0,arrowlength=1.4,arrowinset=0.4]{<-}(2.32,-0.52)(4.36,-0.88)
\usefont{T1}{ptm}{m}{n}
\rput(3.0603125,-1.055){\Large \color{color685}Speech}
\usefont{T1}{ptm}{m}{n}
\rput(3.0803125,1.345){\Large \color{color685}Speech}
\usefont{T1}{ptm}{m}{n}
\rput(6.8231244,2.605){\color{color34}Speech}
\usefont{T1}{ptm}{m}{n}
\rput(11.20875,2.605){\color{color34}Text/Lattice}
\usefont{T1}{ptm}{m}{n}
\rput(12.445,-0.435){\color{color34}Semantic meaning}
\usefont{T1}{ptm}{m}{n}
\rput(11.265624,-2.255){\color{color34}Action}
\usefont{T1}{ptm}{m}{n}
\rput(6.899688,-2.175){\color{color34}Text}
\psline[linewidth=0.04cm](0.8872397,-0.23489322)(0.1,-1.62)
\psline[linewidth=0.04cm](0.36193633,2.0032785)(1.72,2.76)
\psdots[dotsize=0.12](1.34,2.28)
\rput{48.07518}(1.8088533,-0.49558815){\psarc[linewidth=0.04](1.46,1.78){0.12}{0.0}{180.0}}
\psline[linewidth=0.04cm](1.0368464,1.4389827)(1.0031536,1.1610173)
\end{pspicture} 
}

    \caption{Single turn in Alex dialog system}
    \label{fig:dialog_system} 
    \end{center}
\end{figure}

In our dialog system we experiment with different outputs of~a~speech decoder in Spoken Language Understanding unit. 
The decoder for our system should be able to generate {\it n-best lists}, {\it lattices} and {\it confusion networks} output.
% Ideally, the decoder for our system would be able to generate {\it n-best lists}, {\it lattices} and {\it confusion networks}.

%   d) output of~the decoder will be standard lattices 
%     (both phone and word)
%   e) compute posterior lattices (both phone and word),
%   f) provide confusion networks for these lattices
%   g) measure teh quality of~the lattices, depth, oracle error rate    

% section integrate_kaldi_decoder_into_vystadial_framework (end)


% \subsection{Development of~a~real time decoder}
% \label{sub:kaldi_rt_decoder}
% In the our dialog system we need above all a~real time decoder for the Kaldi toolkit. At first, we will identify what prevents the current Kaldi decoders from being used as real time decoders. 
% 
% Namely, we will explore problems with latency and suggest potential speed improvements e.g.\ approximations, use of~\ac{GPU}, and so on. Based on the experiments we will suggest the optimal setting for the real time use of~Kaldi decoder.
% 
% % section development_of_real_time_asr_decoder_for_the_kaldi_toolkit (end)

\subsection{Training acoustic models} 
\label{sub:training_kaldi_acoustic_models}
Every modern continuous speech recognition engine requires two pre trained components, an~acoustic model and a~language model. We will focus on finding the best acoustic model for the~Kaldi toolkit. 

We have been using the \ac{HTK} acoustic models together with OpenJulius and \ac{HTK} decoders. We will develop training scripts for acoustic models using Kaldi toolkit. The acoustic models will be evaluated using different Kaldi decoders and the same language model as used in \ac{HTK} training scripts. 

% section training_kaldi_acoustic_models (end)
 

\subsection{Evaluating decoders} 
\label{sub:compare_rt}
After training acoustic models with~sufficient quality we will balance accuracy and speed for the Kaldi online decoder. Trading decoding speed for a speech recognition accuracy is common technique for speeding up decoders and can be applied on all decoders of interests. We will evaluate following decoders:
\begin{itemize}
    \item HVite  % FIXME test this through Subprocess
    \item HDecode % FIXME test this through Subprocess
    \item OpenJulius % FIXME test this through Subprocess/Sockets
    \item Kaldi decoders
    \item Improved online Kaldi decoder
\end{itemize}

The speed of~the decoders will be expressed in terms of~\ac{RTF}\footnote{\ac{RTF} is a~metric for comparing speed of~decoders. It is defined as $\ac{RTF} = \frac{P}{D}$, where $P$ is time of~the processing the audio by a~decoder on input with length $D$.}. The accuracy will be expressed in terms of~\ac{WER}.

We will also compare and contrast possibilities of~different output formats for each of~tested decoders. The~memory consumption and the~stability of~decoders will be also observed. 

% section compare_real_time_abilities_of_already_used_and_kaldi_decoders (end)

% section goals (end)

\section{Outline of~the thesis} 
\label{sec:outline_of_the_thesis}
In~Chapter~\ref{cha:background} we introduce speech recognition algorithms. At first we introduce standard algorithms. Later, we focus on Finite State Automata~framework used by Kaldi. In~Chapter~\ref{cha:training} we describe how we trained the acoustic models needed by Kaldi decoder. In addition, we shortly contrast and compare training methods in \ac{HTK} and Kaldi. Chapter~\ref{cha:decoder} presents the Kaldi real time decoder embedded into our dialog system. We discuss the architecture of~the decoder, its properties. We also distinguish what is the original work done by Kaldi team and what are our improvements. At the end of~Chapter~\ref{cha:decoder} we describe the qualities of~the developed decoder and compare it with other decoders used in our dialog system.
The~Chapter~\ref{cha:integration} explains how we integrated C++ decoder in our dialog system written in Python.
    
The speech recognition algorithms described in~Chapter~\ref{cha:background} are meant like short introduction to the~problematic, which allow readers understand next chapters. The~Chapter~\ref{cha:training} explains that Kaldi framework is capable of~training acoustic models and decoding with them with no worse accuracy than \ac{HTK} toolkit. In~Chapter~\ref{cha:integration} we discuss technical difficulties related with integration. 

Finally, the~Chapter~\ref{cha:conclusion} summarises the thesis and finishes with future research directions.

% section outline_of_the_thesis (end)
