% !TEX root = main.tex
\chapter{Introduction}
\label{chap:intro}

A spoken dialog is the most intuitive way of~communication among people. 
% Nowadays, the spoken dialog is becoming popular choice of~communication even for human-device interaction. 
The quality of~a~dialog largely depends on the quality 
of~speech recognition, because the~reasoning and the~reply is based on the recognized speech. 

In this work, we prepare \acl{ASR} for a dialog system called Alex. 
We use the Kaldi\cite{povey2011kaldi} speech recognition toolkit for acoustic modeling 
as well as decoding the text transcription from speech using trained \acl{AM}.
We see the added value of this thesis in new training scripts deployed with open acoustic data\cite{korvas_2014}
and in new interface for decoding. Both the decoder interface and acoustic model training is designed 
for a responsive dialog system. 

The Alex dialog system is mostly domain and language independent.
Currently, we are testing the dialog system on a~public Czech phone call service
\footnote{The phone line +420 800 899 998 is for free for Czech callers.}.
A caller can ask about weather and public transport information in Czech republic.

\section{The problem} 
\label{sec:problem}

The speech recognition in a dialog system closely interacts with \acl{SLU} unit.
The \ac{SLU} unit typically classifies the speech better 
if the speech recognizer outputs more than one hypothesis for one utterance. 
A word lattice effectively represent multiple hypothesis and is convenient
representation for passing hypothesis between \ac{ASR} and \ac{SLU} unit 
as described in~Section~\ref{sub:lattice}.

The Alex dialog system has been using the \ac{HTK} toolkit\cite{young94htk} 
to train acoustic models and 
the OpenJulius\cite{lee2009julius} lattice decoder for real time decoding. 
Unfortunately, our project members were experiencing crashes 
of OpenJulius during extracting lattices.

Taking in mind the OpenJulius complicated code base and relatively slow
development of both \ac{HTK} and OpenJulius we were looking 
for another open source toolkit with a real-time decoder.

We choose the~Kaldi toolkit. Despite the fact,  that Kaldi had no lattice real-time decoder
and no interface convenient for a dialog system.
However, the Kaldi toolkit already had modern training recipes.
The~Kaldi toolkit is actively maintained. Moreover, Kaldi is distributed 
under permissive Apache License, Version 2.0\footnote{\url{http://www.apache.org/licenses/LICENSE-2.0}}.

The Kaldi community focused on improving acoustic model training. 
In August 2012\footnote{The changes were introduced by \href{https://sourceforge.net/p/kaldi/code/1259/}{svn commit 1259}.}
Kaldi team published a~demo version of~an online one best hypothesis decoder.
The online decoder inspired us in designing 
the online interface suitable for real-time use of a lattice decoder.
In addition we rewrote the feature preprocessing code base of the one best hypothesis decoder.

\section{The goals of~the~thesis} 
\label{sec:goals}
Let us introduce the goals of~this thesis. Each of~the~goals is described in subsections below.
At first, we introduce the goal of training acoustic model, which is the precondition for satisfying next goals.
Later, we prepare the real-time decoder and at the end, we integrate the decoder in the~dialog system. 

\subsection{Training acoustic models} 
\label{sub:training_kaldi_acoustic_models}
A \acl{CSR} decoder requires two pre trained components, an~\acl{AM} and 
a~\acl{LM}. We focus on finding the best \acl{AM} for the~Kaldi toolkit. 
The \acl{LM} is reused from other experiments.

The \ac{HTK} acoustic models have been used together with OpenJulius. 
We develop acoustic model training scripts using Kaldi toolkit,
so the quality of \ac{AM} is comparable with the~\ac{AM} trained with \ac{HTK} toolkit. 
The scripts are prepared for Czech and English transcribed acoustic data.

% subsection training_kaldi_acoustic_models (end)
 

\subsection{Preparing real-time decoder} 
\label{sub:compare_rt}

We select the most convenient Kaldi decoder and 
transform the decoder to word lattice decoder. 
We suggest potential speed improvements e.g.\ approximations, use of \ac{GPU} 
or \ac{DNN} for speech processing\cite{vesely2013sequencediscriminative}.

The \verb!C++! interface suitable for real-time with a dialog system is designed and implemented.

\subsection{A~real-time decoder integration} 
\label{sub:integration}
The final step is integration of selected real-time decoder into the~dialog system Alex.
The~OpenJulius decoder interfaces the Alex system through Python module, 
which communicates with the decoder through sockets.
The Kaldi is integrated as Python \verb!C++! extension module 
using Cython\footnote{\url{htpp://www.cython.org}}, 
which is not only faster, but also more user friendly solution.

% section goals (end)

\section{Outline of~the thesis} 
\label{sec:outline_of_the_thesis}
In~Chapter~\ref{cha:background} we describe components of a~speech recognition system.  
Except the standard introduction to speech recognition system we introduce Finite State Automata~framework,
which is used by the~Kaldi toolkit. In~Chapter~\ref{cha:training} we describe the acoustic 
models needed by the~Kaldi decoder. 
In addition, we contrast and compare acoustic models trained by~\ac{HTK} and Kaldi. 
Chapter~\ref{cha:decoder} presents the Kaldi real time decoder embedded into our dialog system.
We discuss the architecture of~the decoder, its properties. We also distinguish what is the original work done by 
the~Kaldi team and what are our improvements. At~the end of~Chapter~\ref{cha:decoder} 
we evaluate the qualities of~the developed decoder.
The~Chapter~\ref{cha:integration} explains how we integrated \verb!C++! decoder in our dialog system written in Python.
Finally, the~Chapter~\ref{cha:conclusion} summarises the thesis and finishes with future research directions.

% section outline_of_the_thesis (end)
