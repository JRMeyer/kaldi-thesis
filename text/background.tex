\chapter{Background}
\label{cha:background}

\section{General introduction} 
\label{sec:general_introduction}

\subsection[MFCC coefficients]{Mel-frequency cepstrum coefficients}
\label{sub:mel_frequency_cepstrum_coefficients}



% subsection mel_frequency_cepstrum_coefficients (end)

\subsection{Viterbi algorithm}
\label{sub:viterbi_algorithm}

% subsection viterbi_algorithm (end)

\subsection{Other criteria}
\label{sub:other_criteria}


% subsection other_criteria (end)


% section general_introduction (end)

\section{Finite State Automata} 
\label{sec:finite_state_automata}

\subsection{Description}
\label{sub:description}

% subsection description (end)

\subsection{Kaldi implementation} % (fold)
\label{sec:kaldi}


\subsubsection*{Sources} % (fold)

\begin{itemize}
    \item \href{http://kaldi.sourceforge.net/graph.html} {Decoding graph construction in Kaldi}
    \item \href{http://kaldi.sourceforge.net/lattices.html} {Lattices in Kaldi}
\end{itemize}

\subsubsection*{Decoding graph construction} % (fold)
Kaldi uses Finite-State Transducers (FST) as underlaying representation for all models, which are used to decoding. Consequently, training and decoding of models in Kaldi can be expressed as sequence of operations above FSTs.

Decoding is performed using a final result of training, so called {\it decoding graph}. 
From the high level point of view,
during training we are constructing the decoding graph 
\begin{equation} \label{eq:hclg}
HCLG = H\circ C\circ L\circ G
\end{equation}.

The symbol $\circ$ represents an associative binary operation of composition on FST.
Namely, the transducers appearing in equation \ref{eq:hclg} are:
\begin{enumerate}
    % source  http://kaldi.sourceforge.net/graph.html
    \item G is an acceptor that encodes the grammar or language model.
    \item L is the lexicon. Its input symbols are phones. Its output symbols are words.
    \item C represents the relationship between context-dependent phones on input and phones on output.
    \item H contains the HMM definitions, which takes as input id number of Probability Density function (PDF) and returns context-dependent phones.
\end{enumerate}

Following one liner illustrates how Kaldi creates the decoding graph. 
\begin{equation}
   HCLG = asl(min(rds(det(H' o min(det(C o min(det(L o G)))))))) 
\end{equation}
Let us explain the shortcuts in the list below. Note that the operation are described in detail
at page \href{http://kaldi.sourceforge.net/fst_algo.html#fst_algo_stochastic} {Finite State Transducer algorithms in Kaldi}. 
% The source code of these operations is in fstext and corresponding command-line program are in fstbin/
\begin{itemize}
    \item asl - Add self loops to FST
    \item rds - Remove disambiguation symbols from FST
    \item H' is FST H without self loops
    \item min - FST minimization
    \item $A\circ B$  - Composition of FSTs $A$ and $B$.
    \item det - Determinization of FST
\end{itemize}

{\bf Kaldi stochasticity} - weights of outgoing arcs sum to 1.


\subsubsection*{Kaldi decoders} % (fold)
\begin{itemize}
    \item SimpleDecoder(Beam width) - straightforward implementation of Viterbi algorithm
    \item LatticeSimpleDecoder(Beam width d, Lattice delta $\delta$), where $ \delta \le d$

\end{itemize}
% subsection Kaldi Framework (end)

% section finite_state_automata (end)


% chapter background (end)
