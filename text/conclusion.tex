% !TEX root = main.tex
\chapter{Conclusion}
\label{cha:conclusion}

\todo{The work specified by goals in Chapter~\ref{cha:intro} varies in many aspects. We successfully trained acoustic models, design and also implement decoders interface, improve real-time decoder and prepared experiments for evaluating results.}

The~Kaldi toolkit is a~speech recognition toolkit distributed under a~free license \cite{povey2011kaldi}.
The~toolkit is based on Finite State Transducers, implements state-of-the-art acoustic modelling techniques, is computationally efficient, and is already widely adapted among research groups.
Its only major drawback was the~lack of on-line recognition support.
Therefore, it could not be used directly in applications such as spoken dialogue systems.

This work presented the~\term{OnlineLatgenRecogniser}, an extension of the~Kaldi automatic speech recognition toolkit.
The~\term{OnlineLatgenRecogniser} is distributed under the~Apache 2.0 license, and therefore it is freely available for both research and commercial applications.
The~recogniser and its Python extension is stable and intensively used in a~publicly available spoken dialogue system \cite{ptics2014url}.
Thanks to the~use of a~standard Kaldi lattice decoder, the~recogniser produces high quality word posterior lattices.
% In addition, this enable the~recogniser to benefit from future improvements in the~Kaldi decoders.
% The~scripts for training acoustic models, including Czech data, are also freely available online.
The training scripts for the~acoustic model and the~\term{OnlineLatgenRecogniser} code are currently being integrated in the~Kaldi toolkit.
Future planned improvements include implementing more sophisticated speech parameterisation interface and feature transformations.

% The~\term{OnlineLatgenRecogniser} recogniser, an extension the~Kaldi framework, is freely available\footnote{Apache, 2.0 license} and significantly improved speech recognition for a~dialog system Alex.
% The~recognizer and its Python wrapper is stable and intensively used in a~freely available dialogue system in Czech public transport domain.
% The~speech decoder is able to output high quality lattices, which can be utilised in SLU unit of dialogue system. 
% The~scripts, Czech and English data  used for acoustic training are also freely available online.
% 
% The~\term{OnlLatticeFasterDecoder} modular design allows reusing the~code of \term{LatticeFasterDecoder}, which resulted in improving the~decoder based changes from \term{LatticeFasterDecoder}.
% The~training scripts and also the~speech recognizer code is in process of integration back to Kaldi toolkit.
% 
% \subsection*{Future work}
% Firstly, we se the~future work in implementing more sophisticated feature transformation and speech parameterisation interface, so decoding with larger variety of feature transformations can be supported.
% Secondly, the~recordings with extreme latency values can be detected using the~one-best hypothesis.
% Consequently, the~backward decoding could be skipped for recordings which contains only noise.
% Finally, the~word posterior lattices should be more tightly integrated with SLU unit,
% and the~potential of oracle WER should be more utilised.
% 

% \section*{Acknowledgments}
% This research was partly funded by the~MEYS of the~Czech Republic under the~grant agreement LK11221 and core research funding of Charles University in Prague.
% The work described herein uses language resources hosted by the LINDAT/CLARIN repository, funded by the project LM2010013 of the MEYS of the Czech Republic.
% We would also like to thank Daniel Povey Ondřej Dušek, Matěj Korvas, David Marek and Tomáš Martinec for their useful comments and discussions.
% Ondřej Dušek, Matěj Korvas, David Marek and Tomáš iartinec for useful comments and discussions.  

% include your own bib file like this:

\todo{create future work subsection}
